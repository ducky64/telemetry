\documentclass[border=10pt,png]{article}
\usepackage{bytefield}
\usepackage{color}

\begin{document}

\section{Introduction}

\textcolor{red}{WARNING! This spec is still in BETA and is subject to change.}


\subsection{Big Ideas}
Each piece of data is assigned an ID

\subsection{General Notes}
All multi-byte words are defined to be in network ordering (RFC 1700, big endian).

\section{Packet Layer}
The packet layer structures data provided by the underlying link layer into an organized packet. Currently, there is only one implementation of this layer.

\subsection{In-band Signaling over UART Serial Terminal}
This provides embedding telemetry data in a UART serial terminal stream. It is assumed that no ASCII control characters will be transmitted on the same stream as they will be used to delimit telemetry data.

While not required by the spec, the telemetry transmitter may preface each telemetry frame with ANSI commands to prevent serial terminals from displaying the telemetry (gibberish) contents. ANSI conceal and invisible may be used.

To prevent terminals from interpreting the telemetry data as ANSI commands, anytime an ANSI escape character (\texttt{0x1b}) appears in the data stream, a null character (\texttt{0x00}) should be added directly afterwards. This should be handled by the packet layer only - the byte will be stuffed at this layer at the transmitter and destuffed at the corresponding layer at the receiver. The data length should not include stuffed bytes.

\begin{bytefield}{32}
  \bitheader{0, 15, 16, 31} \\
  \bitbox{16}{Start of Frame \\ \tiny{0x05 0x39}}
  & \bitbox{16}{Data Length \\ \tiny{(2 bytes, number of bytes in the data)}} \\
  \wordbox[lrt]{1}{Data \\ \tiny{payload of bytes equal to length above}} \\
  \skippedwords \\
  \wordbox[lrb]{1}{} \\
  \bitbox{16}{CRC (unimplemented)}
\end{bytefield}

\section{Telemetry Protocol Layer}

Each telemetry packet is structured as follows:

\begin{bytefield}{16}
  \bitheader{0, 7, 8, 15} \\
  \bitbox{8}{Opcode}
  & \bitbox{8}{Sequence \#} \\
  \wordbox[lrt]{1}{Data \\ \tiny{length dependent on opcode}} \\
  \skippedwords \\
  \wordbox[lrb]{1}{}
\end{bytefield}

The sequence number is incremeneted by 1 per transmitted packet and should start at zero. This is used to detect dropped packets.

\subsection{Data format for opcode 0x81: Data Definition}
This is sent to the telemetry client to configure the display. This should be the first telemetry packet sent and should only be sent once (at the start).

\begin{bytefield}{16}
  \bitheader{0, 7, 8, 15} \\
  \bitbox{8}{\# Records} \\
  \wordbox[lrt]{1}{Data headers} \\
  \skippedwords \\
  \wordbox[lrb]{1}{}
\end{bytefield}

Each data header is defined as:

\begin{bytefield}{16}
  \bitheader{0, 7, 8, 15} \\
  \bitbox{8}{Data ID}
  \bitbox{8}{Data type} \\
  \bitbox{8}{Name length} \\
  \wordbox[lrt]{1}{Name characters \\ \tiny{ASCII, non null terminated}} \\
  \skippedwords \\
  \wordbox[lrb]{1}{} \\
  \bitbox{8}{\# KV records} \\
  \wordbox[lrt]{1}{KV records} \\
  \skippedwords \\
  \wordbox[lrb]{1}{}
\end{bytefield}

Each KV record is defined as:

\begin{bytefield}{16}
  \bitheader{0, 7, 8, 15} \\
  \bitbox{8}{Record ID} \\
  \wordbox[lrt]{1}{Record value} \\
  \skippedwords \\
  \wordbox[lrb]{1}{} \\
\end{bytefield}

Record ID meanings are different for each data type. The length and format of the record value is dependent on the record meaning.

\subsection{Data format for opcode 0x01: Data}

\begin{bytefield}{16}
  \bitheader{0, 7, 8, 15} \\
  \bitbox{8}{Data ID} \\
  \wordbox[lrt]{1}{Data value} \\
  \skippedwords \\
  \wordbox[lrb]{1}{} \\
\end{bytefield}

The data value length and format is dependent on the data type, which is defined by the data ID in the header.

\section{Data Types}

\subsection{Int: Data type 0}
\subsubsection{KV Records}
Record ID 0: data length (in bytes)
\subsubsection{Data format}
Raw data in big endian format.

\subsection{Float: Data type 1}
\subsubsection{KV Records}
Record ID 0: data length (in bytes)
\subsubsection{Data format}
Raw data in network order.

\end{document}
